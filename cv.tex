\documentclass[11pt,a4paper,sans]{moderncv}
\usepackage[scale=0.75]{geometry} % Adjust the page layout
\usepackage{xcolor} % Import xcolor package for custom colors

% Define custom burgundy color
\definecolor{color1}{rgb}{0.6435, 0.5430, 0.520}
\colorlet{color1}{color1}
% \moderncvcolor{color1} % Set the color to burgundy
\moderncvstyle{classic} % Style options: classic, banking, casual, etc.
\usepackage[utf8]{inputenc}
% Personal Information
\name{Rudresh}{Gade}
\title{BS-MS Student at IISER Pune} % Title
\address{IISER Pune, Dr.Homi Bhabha Road, Pune, Maharashtra}{India} % Address
\phone[mobile]{+91-9518327069} % Phone number
\email{rudresh.gade@students.iiserpune.ac.in} % Email address
\social[linkedin]{https://www.linkedin.com/in/rudreshgade} % LinkedIn or Personal website

\begin{document}
\makecvtitle
\vspace{4mm}

% About Me Section
\section{About Me}
I am a motivated and research-oriented undergraduate student at IISER Pune, with a strong academic background and a deep passion for Number Theory, Graph Theory, and Spectral Analysis. My fascination with Number Theory arose from exploring numerical patterns and their surprising connections to various fields. Through research internships and academic projects, I have honed my analytical and problem-solving skills, further fueling my interest in advanced mathematical research. I have also explored algebraic graph theory, broadening my interests in this field. I am eager to pursue a Master’s degree to deepen my expertise and contribute to innovative research.
\vspace{4mm}

% Education Section
\section{Education}
\cventry{2021--2026}{Bachelor and Master Dual Degree (BS-MS)}{IISER Pune}{India}{}{CGPA: 7.9}
\cventry{2021}{Senior Secondary (Class 12)}{Shri Shivaji Science Junior College}{Nagpur, India}{}{Percentage: 92.2\%}
\cventry{2019}{Matriculation (Class 10)}{Somalwar Nikhalas High School}{Nagpur, India}{}{Percentage: 96.4\%}
\vspace{4mm}
% Research Projects and Experience Section
\section{Research Projects and Experience}
\cventry{August--Present 2024}{Analytic Number Theory}{IISER Pune}{India}{}{Semester Project supervised by Prof. Kaneenika Sinha, focused on the analytic analysis of Number Theory.}
\cventry{May--Present 2024}{Spectral Graph Theory}{IISER Pune}{India}{}{Research on algebraic connectivity of graphs and its applications, under the guidance of Prof. Anisa Chorwadwala.}
\cventry{May--August 2024}{Homogeneous and Non-Homogeneous Complementary Beatty Sequences}{Smith University}{USA}{}{Worked on the generalization of non-homogeneous complementary sequences under the guidance of Prof. Geremias Polanco.}
\cventry{Jan--April 2024}{Spectral and Algebraic Graph Theory}{IISER Pune}{India}{}{Semester project supervised by Prof. Chandrasheel Bhagwat, focused on the applications of Spectral Graph Theory.}
\cventry{May--Aug 2023}{Stochastic Partial Differential Equations}{VNIT Nagpur}{India}{}{Investigated stochastic PDEs and their behavior under Prof. Naga Raju Gande.}
\vspace{4mm}

% Workshops and Conferences Section
\section{Workshops and Conferences}
\cventry{Jan 2024}{Conference on Combinatorial Game Theory}{IIT Bombay}{India}{}{Attended Conference on Combinatorial games and their applications in combinatorial mathematics.}
\vspace{4mm}

% Relevant Courses Section
\section{Relevant Courses}
\begin{minipage}[t]{0.45\textwidth}
    \textbf{Mathematics:}
    \begin{itemize}
        \item Functional Analysis
        \item Probability
        \item Calculus of Manifolds
        \item Measure Theory
        \item Complex and Real Analysis
        \item Graph Theory
        \item Point Set Topology
        \item Ordinary Differential Equations
        \item Field and Galois Theory
        \item Rings and Modules
        \item Advanced Linear Algebra
        \item Group Theory
    \end{itemize}
\end{minipage}\hfill
\begin{minipage}[t]{0.45\textwidth}
    \textbf{Physics:}
    \begin{itemize}
        \item Gravitation
        \item Non-Linear Dynamics
        \item Group Theory in Physics
        \item Classical Mechanics
        \item Statistical Mechanics
        \item Thermodynamics
        \item Quantum Physics
    \end{itemize}
\end{minipage}
\vspace{4mm}

% Technical Skills Section
\section{Technical Skills}

\cvitem{Scientific Writing}{LaTeX, Microsoft Word}
\cvitem{Programming Languages}{Python, MATLAB}
\cvitem{Operating Systems}{Windows, Linux}
\cvitem{Presentations}{Microsoft PowerPoint}
\vspace{4mm}

% Honors and Awards Section
\section{Honors and Awards}
\cvitem{2021}{INSPIRE Fellowship (Govt. of India): Awarded by the Department of Science and Technology for outstanding academic potential.}
\cvitem{2021}{IISER Aptitude Test (AIR: 129): Secured top rank in the IISER entrance exam.}
\cvitem{2021}{JEE Advanced (AIR: 3913): Ranked in India's top engineering entrance examination.}
\cvitem{2021}{JEE Mains (AIR: 6105): Ranked in India's top engineering entrance examination.}
\vspace{4mm}
\newpage
% Extracurricular Activities Section
\section{Extracurricular Activities}
\cventry{2023}{Integration Bee Organiser}{IISER Pune}{India}{}{Core Team member responsible for organizing and contributing to the success of the Integration Bee competition organised under Maths Club.}
\cventry{2023}{Mimamsa Science Quiz}{IISER Pune}{India}{}{Team member responsible for organizing and contributing to the success of the nationwide science quiz competition.}
\cventry{2022--2023}{DISHA Abhyasika Club Volunteer}{IISER Pune}{India}{}{Led tutoring sessions for underprivileged children, improving their understanding of mathematics and science.}
\cventry{2022--2023}{Vidya Bhavan Teaching Volunteer}{Udaipur}{India}{}{Voluteered teaching class $12^{th}$ Mathematics to the underpreviledged students of rural areas of Udaipur and Chittorgarh in India.}
\vspace{4mm}

% Languages Section
\section{Languages}
\cvitem{English}{Professional working proficiency}
\cvitem{Hindi}{Professional working proficiency}
\cvitem{Marathi}{Native proficiency}
\cvitem{German}{Beginner}
\vspace{4mm}

%Hobbies
\section{Hobbies}
Manga reading(One Piece), watching anime and movies, studying different working principles and application in electronics.  
\end{document}
